\documentclass{beamer}


%&Vary the color applet  (try out your own if you like)
%\colorlet{structure}{red!65!black}

\usetheme{Warsaw}

\usepackage{beamerthemesplit}

\usepackage{ifpdf}          % Detect PDFLaTeX/LaTeX (defines ifpdf)
\usepackage{hyperref}   % For hyperlinks
      
\ifpdf
	\usepackage{graphicx}
	\DeclareGraphicsExtensions{.pdf,.png,.jpg}      % order is important
	%\usepackage[textwidth=1in]{todonotes}

\else
	\usepackage{graphicx}                   % For inclusion of graphics
	\DeclareGraphicsExtensions{.eps}
	%\usepackage[dvistyle,textwidth=1in]{todonotes}

\fi
\graphicspath{{.//figures//}}

\setbeamertemplate{navigation symbols}{}
\usepackage[absolute,overlay]{textpos}
\setlength{\TPHorizModule}{1mm}
\setlength{\TPVertModule}{1mm}
\newcommand{\MyLogo}{%
	\begin{textblock}{14}(2,2)
		\includegraphics[width=2cm]{logo/rit_white_no_bar}
	\end{textblock}
	\begin{textblock}{14}(112,0)
		\includegraphics[width=1.5cm]{logo/netip_logo}
	\end{textblock}
}

\setbeamersize{text margin left=0.1in}
\setbeamersize{text margin right=0.1in}


\usepackage{caption}
\DeclareCaptionFont{tiny}{\tiny}
\captionsetup{labelformat=empty, labelsep=none, font=tiny}

\usepackage{subfigure}


\title[Mathematical Model of the Pacinian Corpuscle]{Mathematical Model of the Pacinian Corpuscle for Design of Haptic Systems}
\author[Daniel Liu]{ Daniel F. Liu \\
		\footnotesize{ Advisor: Dr. Shanchieh Jay Yang }}

\institute[Rochester Institute of Technology]{
	Department of Computer Engineering\\
	Rochester Institute of Technology}
\date[NETIP 2010]{Friday, September 17, 2010 - Research Meeting}
\subject{Computational Neuroscience}




\begin{document}

\frame {
	\MyLogo
	\titlepage
}

\frame {
	\MyLogo
	\tableofcontents
}

\section{Introduction}
\subsection{What is Haptics?}
\frame {
	\MyLogo
	\frametitle{What is Haptics?}


	\begin{itemize}
		\itemsep=.1in
		\item Haptics - the science concerned with the sense of touch/somatic sense.
		\item A part of perceptual science, Haptics can be separated into:
		\begin{itemize}
			\itemsep=.1in
			\item \textbf{Haptic Systems} - tactile feedback systems, tactile HCI
			\item \textbf{Neuroscience} - biophysics, physiology, modeling of somatic system
			\item \textcolor{gray}{\emph{Psychophysics - quantified study for relationships between tactile stimuli and sensation}}
			\item \textcolor{gray}{\emph{Psychology - study of human mental processes}}

		\end{itemize}
	\end{itemize}

}

\subsection{Haptic Systems}
\frame {
	\MyLogo
	\frametitle{Haptic Systems}
	\begin{columns}[T]
	\column{3in}
	\begin{itemize}
		\itemsep=.1in
		\item Haptic Systems (Tactile Feedback Systems)
		\begin{itemize}
			\itemsep=.1in
			\item Tactile Displays
			\begin{itemize}
				\itemsep=.1in
				\item \textbf<1>{Mechanical Pins (Pressure)}
				\item \textbf<2>{Pneumatic Balloons (Pressure)}
				\item \textbf<3>{Electrostatic Resistance (Slip-Shear)}
			\end{itemize}
			\item Tactile Sensors
			\begin{itemize}
				\itemsep=.1in
				\item \textbf<4>{MEMs/Piezoelectric Sensors (Pressure/Vibration)}
				\item \textbf<5>{Peltier Devices (Thermal)}
			\end{itemize}
		\end{itemize}
	\end{itemize}
 
	\column{2in}

	\begin{figure}[c]
		\only<1>{
			\includegraphics[width=1.5in]{"haptic systems"/"2006 - Ottermo - pin tactile display"}
			\caption[]{Pin Array Tactile Display \cite{Ottermo2006}}
		}
		\only<2>{
			\includegraphics[width=1.5in]{"haptic systems"/"2008 - King - pneumatic ballon davinci display"}
			\caption[]{Pneumatic Ballon Tactile Display \cite{Culjat2008a}}
		}
		\only<3>{
			\includegraphics[width=1.5in]{"haptic systems"/"2007 - Yokota - softness electrostatic slip-shear display"}
			\caption[]{Electrostatic Slip-Shear Display \cite{Yokota2007}}
		}
		\only<4>{
			\includegraphics[width=1.5in]{"haptic systems"/"2005 - Valdastri - tri-axial force sensor"}
			\caption[]{Electrostatic Slip-Shear Display \cite{Beccai2005}}
		}
		\only<5>{
			\includegraphics[width=1.5in]{"haptic systems"/"wikimedia - peltier device"}
			\caption[]{Solid-state Thermoelectric Device}
		}
	\end{figure}

	\end{columns}

}

\section{Motivation}
\subsection{Haptic Systems}
\frame {
	\MyLogo
	\frametitle{Motivation}

	\begin{columns}[c]
	\column{3in}
	\begin{itemize}
		\itemsep=.1in
		\item Need for enhanced human computer interaction with new remote technologies:
		\begin{itemize}
			\itemsep=.1in
			\item \textbf<1>{Aircraft Controls}
			\item \textbf<2>{Virtual Reality}
			\item \textbf<3>{Haptic Feedback for Robotics and Medical Devices}

		\end{itemize}
	\end{itemize}
	\vspace{1in}

	\column{2in}
	\begin{figure}[c]
		\only<1>{
			\includegraphics[width=1.5in]{"haptic systems"/"MiG-29 Joystick"}
			\caption[]{Modern Aircrafts with Force Feedback (MiG-29 Joystick)}
		}
		\only<2>{
			\includegraphics[width=1.5in]{"haptic systems"/"Sensable Haptic Rendering"}
			\caption[]{Sensable Force Feedback Rendering Device}
		}
		\only<3>{
			\includegraphics[width=1.5in]{"haptic systems"/"Da Vinci Surgical System"}
			\caption[]{Da Vinci Minimally Invasive Robotic Surgery System}
		}
	\end{figure}
	\end{columns}
}

\subsection{Haptic System Limitations}
\frame {
	\MyLogo
	\frametitle{Haptic System Limitations}
	\begin{itemize}
		\itemsep=.1in
		\item Sensors and Displays have limited functionality
		\begin{itemize}
			\itemsep=.05in
			\item Restricted to a subset of sensory modalities
			\item Limited resolution, accuracy, and size
			\item Signal propagation delay affects perception

		\end{itemize}
		\item Limited understanding in human perception of touch
		\begin{itemize}
			\itemsep=.05in
			\item Modalities and taxonomy of tactile perceptions ill-defined
			\item Physiological processes in somatic system partially known
			\item Perception of touch difficult to quantify
		\end{itemize}
		\item Traditionally few researchers actively study perceptual science with haptic systems

	\end{itemize}
}
\subsection{Problem Statement}
\frame {
	\MyLogo
	\frametitle{Problem Statement}
	\begin{columns}[c]

	\column{3in}
		\begin{itemize}
			\item Mathematical Model of the Pacinian Corpuscle (PC)
			\begin{itemize}
				\item Extend existing models and research
				\begin{itemize}
					\item Frequency Response
					\item \textbf{Thermal Response}
					\item \textbf{Sensory Adapatation}
				\end{itemize}
			\end{itemize}
			\item Responsible for high frequency sensations
			\begin{itemize}
				\item Texture
				\item Fine control of tools
			\end{itemize}
			\item Receptor targeted by vibrotactile haptic displays
			\item \footnotesize Quantify different effects
			\item \footnotesize Discover and support physiogical theories
			\item \footnotesize Offer design requirements or processing models for haptic systems
		\end{itemize}
	\column{2in}
	\begin{figure}[c]
		\includegraphics[width=1.5in]{"PC"/"McGraw Hill - Cutaneous Receptors"}
		\caption[]{Cutaneous Receptors in the skin \copyright McGraw Hill}
	\end{figure}
	\begin{figure}[c]
		\includegraphics[width=1.5in]{"PC"/"1991 - Bell - PC Physiological Schematic"}
		\caption[]{Morphology and Schematic of Pacinian Corpuscle \cite{Holmes1990}}
	\end{figure}
	\end{columns}
	
}

\section{Related Work}
\subsection{Physiological Data}
\frame {
	\MyLogo
	\frametitle{Physiological Data}
	Example Datasets from Physiology Experiments
	\begin{columns}[t]
	\column{1.5in}
	\begin{figure}[t]
		\includegraphics[width=1.5in]{"PC"/"1961 - Sato - RC Vibration frequency vs spike threshold"}
		\caption[]{Stimuli vibration frequency vs relative threshold to generate neural spikes \cite{Sato1961} }
	\end{figure}

	\column{1.5in}
	\begin{figure}[t]
		\includegraphics[width=1.5in]{"PC"/"1982 - bolanowski - PC Intensity Characteristics"}
		\caption[]{PC firing rate over varying stimuli intensities and frequencies \cite{BolanowskiJr1982}}
	\end{figure}
	\column{1.5in}
	\begin{figure}[t]
		\includegraphics[width=1.5in]{"PC"/"1982 - bolanowski - PC Intensity over Thermal Characteristics 30Hz 100Hz"}
		\caption[]{PC firing rate over varying stimuli intensities and temperatures at 30 and 100 Hz \cite{BolanowskiJr1982}}
	\end{figure}
	\end{columns}
}
\subsection{Phenomenological Models}
\frame {
	\MyLogo
	\frametitle{Phenomenological Models}

	\begin{columns}[b]

	\column{2.5in}
	\begin{figure}[b]
		\includegraphics[width=2in]{"PC"/"1982 - Freeman - equivalent circuit model of temporal discharge patterns"}
		\caption[]{4 Parameter Kinetic Model of PC Represented as a Circuit \cite{Freeman1982b}}
	\end{figure}

	\column{2.5in}
	\begin{figure}[b]
		\includegraphics[width=2.5in]{"PC"/"1983 - Grandori- PC equivalence model"}
		\caption[]{Linear Filter and Curve Fitted Equivalent Model \cite{Grandori1983}}
	\end{figure}

	\end{columns}

	\begin{itemize}
		\item Two example phenomenological (equivalence) models based on observed stimuli response and simplified models
		\item Little biological basis for model implementation
		\item Does not model all experimental observations
		\item Not easily extended with parameters not explicitly modeled (e.g. temperature, adaptivity)
	\end{itemize}
}

\subsection{Biophysical Model}

\frame {
	\MyLogo
	\frametitle{Biophysical Model}
	\begin{itemize}
		\item Model by Bell and Holmes \cite{Holmes1990}\cite{Bell1992}
		\begin{itemize}
			\item Extends Hodgkin Huxley Model \cite{Hodgkin1952} of neuron electrical properties
		\end{itemize}
	\end{itemize}

	\begin{columns}[t]

	\column{2.5in}[b]
	\begin{figure}[b]
		\includegraphics[width=2.5in]{"hh"/"wikimedia - action potential"}
		\caption[]{Example electrical signal (action potential) in nerves}
	\end{figure}

	\column{2.5in}
	\begin{figure}[b]
		\includegraphics[width=2.5in]{"PC"/"1991 - Bell - PC Physiological Schematic"}
		\caption[]{Morphology and Schematic of Pacinian Corpuscle \cite{Holmes1990}.  Action potentials start in the unmyelinated terminus of the dendrite and propogate down into the nervous system}
	\end{figure}

	\end{columns}
}

\frame{
	\MyLogo
	\frametitle{Biophysical Model - Hodgkin Huxley}
	\begin{itemize}
		\item Different concentration of ions on each side of membrane
		\item Channels open and close at different voltages
		\item Changing ion concentrations cause current and voltage change (electrical signal)
	\end{itemize}
	\begin{figure}
		\subfigure{\includegraphics[width=2in]{"hh"/"nature - ion channel"}}
		\subfigure{\includegraphics[width=2in]{"hh"/"HH circuit schematic"}}
		\caption[]{\textbf{Left:} Nerve membrane with voltage sensitive, gated ion channels. \textbf{Right:} Circuit representation of Hodgkin Huxley's model of channels.}
	\end{figure}
}

\frame{
	\MyLogo
	\frametitle{Biophysical Model - Cable Theory \& Compartment Model}
	\begin{itemize}
		\item Describes signal propagation through an ideal cylindrical cable
	\end{itemize}
\[\textstyle \frac{1}{r_{i}} \frac{\delta^{2}V_{m}}{\delta x^{2}}=C_{m}*\frac{\delta V_{m}}{\delta t} + \frac{V_{m}}{r_{m}}\]
\scriptsize where:\\
 $r_{i}$ - internal cytoplasm resistance ($\Omega/cm$)\\
 $V_{m}$ - membrane voltage ($V$)\\
 $x$ - length along membrane ($cm$)\\
 $r_{m}$ - membrane resistance ($\Omega/cm$)\\
\normalsize
	\begin{itemize}
		\item Compartment model is discrete representation of cable propagation
	\end{itemize}
	\begin{figure}[b]
		\includegraphics[width=3in]{"hh"/"cable theory circuit"}
		\caption[]{ Compartment Model of Cable Theory in Circuit Representation }
	\end{figure}
}

\frame {
	\MyLogo
	\frametitle{Biophysical Model - Bell \& Holmes}
	\begin{itemize}
		\itemsep=.1in
		\item Models mechanical properties of capsule - translates displacement of corpuscle surface to hoop-strain
		\item Exact transduction method unclear
		\begin{itemize}
			\item Some evidence shows a new strain dependent sodium channel
			\item Hodgkin Huxley Sodium Channel modified to be voltage and strain energy dependent
\end{itemize}
		\item Rigorous mathematical analysis (systems mostly solved analytically)
		\item Qualitative comparisons against experimental data
	\end{itemize}
}

\section{Methodology}
\subsection{Simulation Enviornment}
\frame {
	\MyLogo
	\frametitle{NEURON}
	\begin{itemize}
		\item NEURON Simulator (Yale University)
		\begin{itemize}
			\itemsep=.1in
			\item Well supported computational neural simulator (500+ models over 15 years)
			\item Compartmental model
			\item Solves arbritary dynamical systems in each compartment using a backward Euler or Crank-Nicholson solver
			\item ideal for implemental channel models in custom language NMODL
			\item No published mechanoreceptor models

\end{itemize}
\end{itemize}
}

\subsection{New Parameters}
\frame {
	\MyLogo
	\frametitle {Thermal Parameters}
	\begin{enumerate}
		\item All Nernst (e.g. $E_{K}, E_{Na}$) potentials are based on temperature. (e.g. $E_{K}=-\frac{RT}{nF}ln\frac{[K]_{in}}{[K]_{out}}$)

		\item The time/rate constants $\alpha$ and $\beta$ changes based on a proportional relationship with $Q$.
		\[Q=3^{[t-6.3]/10}\]
		where for example:
		\[\frac{dn}{dt}=Q\alpha_{n}[1-n]-Q\beta_{n}n\]

		\item The maximum conductance of an ion (e.g. $N_{k}\gamma_{k}, N_{Na}\gamma_{Na}$) also depends on temperature, which is not usually considered in H-H simulations because is not as significant as other parameters.  May be significant in the PC's initial gradient response.
	\end{enumerate}
}

\frame {
	\MyLogo
	\frametitle{Sensory Adaptation}
	\begin{itemize}
		\itemsep=.1in
		\item Rapid Adaptation - caused by mechanical high-pass filter behavior of corpuscle
		\item Evidence shows tactile receptors could have multiple ion channels with different time constants contributing to different rates of adaptation \cite{French1994}
		\item Neurotransmitter feedforward could contribute to adaptation

	\end{itemize}
}



\frame {
	\MyLogo
	\frametitle{Channels for Mechanoreception}
	\begin{columns}[c]
	\column{2in}
	Three possible types activation methods for mechanoreceptors. Experimental data supports stretch (a) and tethered (b) type channels, while indirect gating (c) is hypothesized by some researchers.
	\column{2in}
	\begin{figure}
		\includegraphics[width=1.8in]{"mechanoreceptor"/"2007 - Lumpkin - mechanoreceptor channel types"}
	\end{figure}
	\end{columns}
}


\frame {
	\MyLogo
	\frametitle{Channels for Mechanoreception}

	Table of known tactile (mechanical and thermal) channels in mammals \cite{Lumpkin2007}\cite{Belmonte2008}
	\tiny

	\begin{tabular}{|l|l|l|}
		\hline
		\textbf{Channel Name} & \textbf{Modalities} & \textbf{Found In} \\
		\hline
		\hline
		TRPA1 & Thermal, mechanical & C-fibres \\
		\hline
		TRPC1 & Mechanical & Mechanoreceptors \\
		\hline
		TRPM8 & Thermal, mechanical (membrane tension) & C-fibres \\
		\hline
		TRPV1 & Thermal, osmotic, mechanical (modulatory) & C-, A(delta)-fibres, keratinocytes \\
		\hline
		TRPV2 & Thermal, osmotic, mechanical & A(delta)-, A(beta)-fibres, immune cells \\
		\hline
		TRPV4 & Thermal, osmotic, mechanical (sheer stress) & Keratinocyptes, Merkel cells, A(delta)-, C-fibres \\
		\hline
		ASIC1 & Mechanical (?) , cold (positive modulator) & A(delta)-, A(beta)-, C-fibres \\
		\hline
		ASIC2 & Mechanical & A(delta)-, A(beta)-fibres \\
		\hline
		ASIC3 & Mechanical, cold (positive modulator), nociception & A(delta)-, A(beta)-fibres \\
		\hline
		TREK-1 & Thermal (heat), mechanical & A(delta)-, A(beta)\- (?) , C-fibres \\
		\hline
		TREK-2 & Thermal (heat), mechanical & \\
		\hline
	\end{tabular}
}

\subsection{More Theory/Implementation}
\frame {
	\MyLogo
	\frametitle{Original Hodgkin Huxley Model}
	\[\scriptstyle I_{m} = I_{k}+I_{Na}+I_{L}+I_{c}\]
	\[\scriptstyle I_{K} = g_{K}(t,v_{m})[V_{m}-E_{K}]\]
	\[\scriptstyle g_{K}(t,v_{m})=N_{K} \gamma_{Na} n^{4}(t,v_{m})\]
	\[\scriptstyle I_{Na} = g_{Na}(t,v_{m})[V_{m}-E_{Na}]\]
	\[\scriptstyle  g_{Na}(t,v_{m})=N_{Na} \gamma_{Na} m^{3}(t,v_{m})h(t,v_{m})\]
	\[\scriptstyle I_{L}=g_{L}[V_{m}-E_{L}]\]
	\[\scriptstyle I_{c}=C_{m}\frac{d V_{m}}{dt}\]
	\tiny where \\
	K - Potassium\textsuperscript{+} ion, Na - Sodium\textsuperscript{+} ion, L - Leakage, c - Capacitance \\
	$N_{x}$ - number of x-ion channels (or number of x-ion channels/area) \\
	$V_{m}$ - membrane voltage \\
	$E_{x}$ - Nernst Potential x-ion \\
	$g_{x}$ - proportionality coefficient for x-ion (macroscale conductance) \\
	$\gamma_{x}$ - conductance of x-ion channel \\
	$N_{x} \gamma_{x}$ - maximum conductance of N x-ion \\
	$m^{3}$ - probability of 3 activation subchannels for Na \\
	$h$ - probability of 1 inactivation subchannel for Na \\
	$n^{4}$ probability of 4 activation subchannels for K
}

\frame {
	\MyLogo
	\frametitle{Channel Equations}
		\[\scriptstyle \frac{dn(t,v_{m})}{dt} = \alpha_{n}(v_{m})[1-n]-\beta_{n}(v_{m})n\]
		\[\scriptstyle \frac{dm}{dt} = \alpha_{m}(v_{m})[1-m]-\beta_{m}(v_{m})m\]
		\[\scriptstyle \frac{dh}{dt} = \alpha_{h}(v_{m})[1-h]-\beta_{h}(v_{m})h\]
		
		Original Hodgkin Huxley parameters
		\[\scriptstyle \alpha_{n}=\frac{0.01(10-v_{m})}{exp(\frac{10-v_{m}}{10})-1} \qquad \beta_{n}=\frac{1}{exp(8 \frac{v_{m}}{80})} \]
		\[\scriptstyle \alpha_{m}=\frac{0.1(25-v_{m})}{exp(0.1[25-v_{m}])-1} \qquad \beta_{m}=\frac{1}{0.25exp(\frac{v_{m}}{18})} \]
		\[\scriptstyle \alpha_{h}=\frac{0.01(10-v_{m})}{exp(\frac{10-v_{m}}{10})-1} \qquad \beta_{h}=\frac{0.125}{exp(\frac{v_{m}}{80})} \]
		
		New Sodium parameter for Bell \& Holmes
		\[\scriptstyle \alpha_{m}(v_{m},U)=0.000792027(T+273.15)exp[a_{1}U+a_{2}v_{m}] \]
}

\frame {
	\frametitle{Cable Theory with Excitable Spines}
	\MyLogo
	\begin{itemize}
		\item Improved cable theory model for dendrites with spines (numerous extensions of the membrane that may contain ion channels)
	\end{itemize}
	\[ \pi d C_{m} \frac{\delta V_{m}}{\delta t} = \frac{\pi d^{2}}{4 R_{i}} \frac{\delta^{2} V_{m}}{\delta x^{2}} - \frac{\pi d}{R_{m}} V_{m} + \bar N(x) I_{ss}\]

	\begin{columns}[t]
	\column{1.5in}
	\tiny where: \\
	 $C_{m}$ - membrane capacity ($\mu F/cm^{2}$) \\
	 $R_{i}$ - cytoplasmic/internal resistivity ($\Omega \cdotp cm$)\\
	 $R_{m}$ - membrane resistance ($\Omega \cdotp cm^{2}$) \\
	 $d$ - diameter of axon (cm)\\
	 $V_{m}$ - membrane voltage\\
	 $I_{ss}$ - current from single spine\\
	 $\bar N(x)$ - density of spines (spines/unit length)\\
	$\Delta x \bar N(x) I_{ss}$ - current delivered over length $\Delta X$ by spines\\

	\column{3in}
	\begin{figure}[b]
		\includegraphics[width=3in]{"hh"/"cable theory excitable spine circuit"}
		\caption[]{ Extended Cable Theory with Excitable Spines (Compartment Circuit Representation) }
	\end{figure}
	\end{columns}
}

\frame[allowframebreaks]{
	\MyLogo
	\bibliographystyle{IEEEtrans}
	\tiny{\bibliography{library}}
}
\end{document}
