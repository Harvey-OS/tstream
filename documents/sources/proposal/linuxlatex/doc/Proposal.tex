%%%%%%%%%%%%%%%%%%%%%%%%%%%%%%%%%%%%%%%%%%%%%%%%%%%%%%%%%%%%%%%%%%%%%%%%%%%%%%%
%
% THESIS DESCRIPTION:
%   A concise description of the main concepts of the thesis.
%
% RESEARCH:
%   A list of research activities which led to this thesis.
%
% EXPERIMENTS:
%   A list of the experiments performed which supported the research.
%
%%%%%%%%%%%%%%%%%%%%%%%%%%%%%%%%%%%%%%%%%%%%%%%%%%%%%%%%%%%%%%%%%%%%%%%%%%%%%%%
\documentclass[12pt,american]{report}
\usepackage{rit-proposal}
%%%%%%%%%%%%%%%%%%%%%%%%%%%%%%%%%%%%%%%%%%%%%%%%%%%%%%%%%%%%%%%%%%%%%%%%%%%%%%%
%   The following packages are all optional and depend on the specifics of what
% is contained in the proposal.  There is no harm in leaving them in.
%%%%%%%%%%%%%%%%%%%%%%%%%%%%%%%%%%%%%%%%%%%%%%%%%%%%%%%%%%%%%%%%%%%%%%%%%%%%%%%
\usepackage{subfigure}
\usepackage{babel}
\usepackage{times}
\usepackage{graphicx}
\usepackage{amssymb}
\usepackage{lscape}
\usepackage{verbatim}
\usepackage{enumerate}
\usepackage{afterpage}
\usepackage{longtable}
\setlongtables

%%%%%%%%%%%%%%%%%%%%%%%%%%%%%%%%%%%%%%%%%%%%%%%%%%%%%%%%%%%%%%%%%%%%%%%%%%%%%%%
%   Mark the document as 'draft' with a date. Be sure to comment this out for
% the final version.
%\usepackage{watermark}
%\watermark{\hspace{-0.3in} \textbf{DRAFT} \hspace{2.0in} \textbf{\today}}
%%%%%%%%%%%%%%%%%%%%%%%%%%%%%%%%%%%%%%%%%%%%%%%%%%%%%%%%%%%%%%%%%%%%%%%%%%%%%%%

\begin{document}
%%%%%%%%%%%%%%%%%%%%%%%%%%%%%%%%%%%%%%%%%%%%%%%%%%%%%%%%%%%%%%%%%%%%%%%%%%%%%%%
% Title page
% The \title{} can contain line breaks as appropriate...
\title{The Proposed Thesis Title}
% The \titleline{} must have no line breaks in it.
\titleline{The Proposed Thesis Title}
% There should be no reason to change the \thesistype{} or the \MSThesistrue...
\thesistype{Thesis Proposal}
\MSthesistrue
% This date is really not used (unless \grantdate{}{} is blank)
\date{July 2004}
%%%%%%%%%%%%%%%%%%%%%%%%%%%%%%%%%%%%%%%%%%%%%%%%%%%%%%%%%%%%%%%%%%%%%%%%%%%%%%%

%%%%%%%%%%%%%%%%%%%%%%%%%%%%%%%%%%%%%%%%%%%%%%%%%%%%%%%%%%%%%%%%%%%%%%%%%%%%%%%
% Author information page
% The \author{} should be exactly the same as your diploma
\author{Your Name (as it will appear on your diploma)}
%%%%%%%%%%%%%%%%%%%%%%%%%%%%%%%%%%%%%%%%%%%%%%%%%%%%%%%%%%%%%%%%%%%%%%%%%%%%%%%

%%%%%%%%%%%%%%%%%%%%%%%%%%%%%%%%%%%%%%%%%%%%%%%%%%%%%%%%%%%%%%%%%%%%%%%%%%%%%%%
% The following information is for the signature page.
% Note that the definition for principal adviser uses two fields.
% This was needed so that the adviser's name could be placed on the
% abstract page without his/her title.
% \foursigstrue | \fivesigstrue but don't define BOTH to be true!!
\principaladviser{Primary Adviser}{Primary Adviser Title}
\firstreader{Secondary Adviser One \\ Secondary Adviser Title and Department}
\secondreader{Secondary Adviser Two \\ Secondary Adviser Title and Department}
% Use this only if \foursigstrue
%\thirdreader{Reader Three \\ Reader3 Title}
%\thirdreader
% Use this only if \fivesigstrue
%\fourthreader{Reader Four \\ Reader4 Title}
%%%%%%%%%%%%%%%%%%%%%%%%%%%%%%%%%%%%%%%%%%%%%%%%%%%%%%%%%%%%%%%%%%%%%%%%%%%%%%%

%%%%%%%%%%%%%%%%%%%%%%%%%%%%%%%%%%%%%%%%%%%%%%%%%%%%%%%%%%%%%%%%%%%%%%%%%%%%%%%
% This is the expected date that the committee will sign your proposal.
\grantdate{July}{2004}
%%%%%%%%%%%%%%%%%%%%%%%%%%%%%%%%%%%%%%%%%%%%%%%%%%%%%%%%%%%%%%%%%%%%%%%%%%%%%%%

%%%%%%%%%%%%%%%%%%%%%%%%%%%%%%%%%%%%%%%%%%%%%%%%%%%%%%%%%%%%%%%%%%%%%%%%%%%%%%%
% If you want to copyright your thesis / dissertation remove the line below.
\copyrightfalse% True by default
% The year of the copyright; usually same as the date the committee will
% sign the thesis. This won't be printed if \copyrightfalse
\copyrightyear{2004}
%%%%%%%%%%%%%%%%%%%%%%%%%%%%%%%%%%%%%%%%%%%%%%%%%%%%%%%%%%%%%%%%%%%%%%%%%%%%%%%

%%%%%%%%%%%%%%%%%%%%%%%%%%%%%%%%%%%%%%%%%%%%%%%%%%%%%%%%%%%%%%%%%%%%%%%%%%%%%%%
% This causes all front matter to be set.
\beforepreface%
%%%%%%%%%%%%%%%%%%%%%%%%%%%%%%%%%%%%%%%%%%%%%%%%%%%%%%%%%%%%%%%%%%%%%%%%%%%%%%%

%%%%%%%%%%%%%%%%%%%%%%%%%%%%%%%%%%%%%%%%%%%%%%%%%%%%%%%%%%%%%%%%%%%%%%%%%%%%%%%
% Set to double spaced
\renewcommand{\baselinestretch}{1.5}
\small\normalsize
%%%%%%%%%%%%%%%%%%%%%%%%%%%%%%%%%%%%%%%%%%%%%%%%%%%%%%%%%%%%%%%%%%%%%%%%%%%%%%%

%%%%%%%%%%%%%%%%%%%%%%%%%%%%%%%%%%%%%%%%%%%%%%%%%%%%%%%%%%%%%%%%%%%%%%%%%%%%%%%
%%  Collection of useful abbreviations.
\newcommand{\etc} {\emph{etc.\/}}
\newcommand{\etal}{\emph{et~al.\/}}
\newcommand{\eg}  {\emph{e.g.\/}}
\newcommand{\ie}  {\emph{i.e.\/}}
%%%%%%%%%%%%%%%%%%%%%%%%%%%%%%%%%%%%%%%%%%%%%%%%%%%%%%%%%%%%%%%%%%%%%%%%%%%%%%%

%%%%%%%%%%%%%%%%%%%%%%%%%%%%%%%%%%%%%%%%%%%%%%%%%%%%%%%%%%%%%%%%%%%%%%%%%%%%%%%
% Abstract
\begin{abstractpage}
The abstract should \emph{stand alone} in the sense that the reader should be 
able to get a good idea of the proposal from reading only the abstract.  An 
abstract never contains references (citations) to other work since the same 
material presented in the abstract will be presented (in a more in-depth 
manner) in the main body of the text (typically the Introduction section).

Three paragraphs are generally sufficient, following the "tell the reader 
what you will say, say it, and tell the reader what you said" format.

The last paragraph generally provides a description of the expected 
outcome and the deliverables (i.e., the intellectual and/or physical 
product that you expect to produce).
\end{abstractpage}
%%%%%%%%%%%%%%%%%%%%%%%%%%%%%%%%%%%%%%%%%%%%%%%%%%%%%%%%%%%%%%%%%%%%%%%%%%%%%%%

%%%%%%%%%%%%%%%%%%%%%%%%%%%%%%%%%%%%%%%%%%%%%%%%%%%%%%%%%%%%%%%%%%%%%%%%%%%%%%%
% Leave these alone for the proposal...
% Uncomment the line below if you don't want a list of tables to be printed.
\tablespagefalse

% Uncomment the line below if you don't want a list of figures to be printed.
\figurespagefalse

% \afterpreface generates the table of contents, list of tables (optional),
% and list of figures (optional).
%\afterpreface%
%%%%%%%%%%%%%%%%%%%%%%%%%%%%%%%%%%%%%%%%%%%%%%%%%%%%%%%%%%%%%%%%%%%%%%%%%%%%%%%

%%%%%%%%%%%%%%%%%%%%%%%%%%%%%%%%%%%%%%%%%%%%%%%%%%%%%%%%%%%%%%%%%%%%%%%%%%%%%%%
% This is where the main body of the proposal starts
\body%
\chapter{Thesis Objectives}
This section essentially goes into more depth regarding the work you 
expect to do and, more importantly, why you need to do it.  You should 
be building a case to justify doing the proposed work as well as 
describing it.  The idea is that not only do your committee members need 
to understand what you are proposing but they also need to understand that 
you have a solid understanding of where this works fits in relation to other 
work in this area that has already been completed.  This is important for a 
number of reasons: 1) You cannot propose as an MS thesis that you re-do the 
work that someone else has already done (there is no intellectual merit in 
that exercise and it is not worthy of an MS Thesis - as your primary adviser 
should have explained to you) and 2) Failure to understand the "landscape" 
of your thesis will likely mean that you will not take full advantage of 
prior results, and therefore your research will not have as significant 
an impact as it might otherwise have.

\section{Supporting Work}
In this section (a.k.a., "Literature Search") you provide proof that you 
have thoroughly studied the problem and the proposed solution.  This is 
accomplished by summarizing the relevant research papers in your area 
and explaining how your work goes beyond what has been done (by anyone) 
in some demonstrable manner.  This is not a section where you simply list 
papers with a brief summary of the contents.  The idea is to weave a story 
from the prior work explaining the path that research has followed.  What 
you are trying to do in this section is lead the reader to the conclusion 
that your proposal is a natural extension of the work that has already 
been done.  This is probably the largest section, but keep in mind  
everything that you write in your proposal is 100\% re-usable in your thesis.

\section{Project Deliverables}
In this section you explicitly describe the things that do not exist now 
but which will exist when you are finished.  This often takes the form 
of a list, for example:
\begin{itemize}
\item The thesis document itself
\item A white paper that summarizes this work, to be submitted for publication
\item ...Other things that you will "produce"...
\end{itemize}

\chapter{Thesis Outline}
This section is simply an outline of your thesis document.  The idea is to 
show that you have thought about the overall construction of your argument 
and how you will present it.  It is an \emph{argument} because you have to 
support each claim in the thesis with facts either from someone else's 
work (in which case you provide the appropriate reference) or from your 
experiment(s) (in which case you provide the details of the experiments 
and the results).  The construction of the thesis is one where you 
1) Describe the problem; 2) Provide motivation (i.e., why is this problem 
important); 3) Provide background material (i.e., what work has come 
before yours and how does yours fit in with all of that work); 4) Describe 
your solution; 5) Describe your evaluation method(s); 6) Present your 
results; 7) Summarize and opine regarding the direction of future work 
(future work beyond the thesis that (presumably) someone else will perform).

\chapter{Schedule}
This section describes how you see the work proceeding.  This is necessarily 
course and inherently inaccurate (everyone knows that so don't worry).  
You can include a Gantt Chart or a simple two-column table with tasks and 
dates.  It is probably best to stick with weeks as your time scale; at this 
point, anything more fine-grain than that doesn't make much sense (the 
argument can be made that for some aspects of the schedule months is a 
more appropriate granularity -- that is something that you work out with 
your primary adviser).

\chapter{Required Resources}
This section will vary significantly from thesis to thesis.  The section needs 
to be here even if you do not need any resource(s).  The idea is to make it 
clear what you will need in order to complete the schedule as you have defined 
it.  That is, you may need a workstation in a lab, or you may need specialized 
software, or you may need special access to a computer or piece of equipment.  
Whatever it is, you need to provide all the information (including dates, if 
appropriate) so that you and your primary adviser can ensure that you have 
what you need.  In addition to providing the details of the resource(s) you 
should provide a brief justification which directly ties into your thesis work.

% Just to get the bibliography to show up...delete it.
\nocite{*}
%%%%%%%%%%%%%%%%%%%%%%%%%%%%%%%%%%%%%%%%%%%%%%%%%%%%%%%%%%%%%%%%%%%%%%%%%%%%%%%

%%%%%%%%%%%%%%%%%%%%%%%%%%%%%%%%%%%%%%%%%%%%%%%%%%%%%%%%%%%%%%%%%%%%%%%%%%%%%%%
\bibliographystyle{plain}
% Delete this text after you have read it...
%In this section you provide a properly formatted bibliography for all papers 
%(and only those papers) directly cited in the proposal.  Your committee has 
%the right to expect that you have read everything that you cite (this holds 
%true for the thesis as well).  Under no circumstances should you include 
%papers in your proposal/thesis which you have not read and which you do 
%not understand.  If you have constructed your arguments properly, this 
%should not be a problem.  The format of the bibliography should be the 
%same used in technical paper publications and it should include all 
%information necessary for the reader to get the source document.  Keep 
%in mind that only in special circumstances (e.g., product specifications 
%or data sheets) should URLs be used as references.
% Single space the bibliography to save space.
\begin{singlespace}
\bibliography{Proposal}
\end{singlespace}
%%%%%%%%%%%%%%%%%%%%%%%%%%%%%%%%%%%%%%%%%%%%%%%%%%%%%%%%%%%%%%%%%%%%%%%%%%%%%%%
\end{document}
